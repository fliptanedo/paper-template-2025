% BUT: REQUIRES XELATEX
%!TEX root = ../sample_lecturenotes.tex
%% Update the above with the appropriate root

\chapter{Margins and Figures} %% if using report class
% \chapterminitoc

The main difference between my lecture notes and my paper template is the large margin to make use of side notes. The margin is a natural place to encourage students to jot notes and to host useful marginalia.%
\begin{marginfigure}%[th]
    \includegraphics[width=.8\textwidth]{example-image-golden}
    \captionsetup{font={scriptsize,sf}}
    \caption{Example of a margin figure.}
    \label{fig:figure:example:golden}
\end{marginfigure}
You can even place figures in the margin, see Fig.~\ref{fig:figure:example:golden}.\sidenote{Look! No float collision!}

\section{Vision}

This document is inspired by Edward Tufte and the implementation of his ideals in the \texttt{tufte-latex} package. The notes have a large margin for side notes and floats (e.g.\ figures).  Visually this means that the column of main text is narrower, which permits a slightly smaller font.


\section{Using the Margin}

We implement marginalia with the \texttt{sidenotes} package\index{side notes}. We highlight the main usage here. A standard sidenote uses \verb!\sidenote{...}! and looks like this\sidenote{Test of a sidenote. Let's add some extra text here to demonstrate the following point about non-overlapping notes}. Unlike \verb!marginnote!s, \verb!sidenotes! do not overlap with each other.\sidenote{An example of a sidenote that does not overlap with the previous one.} What more, the sidenotes coexist fine with footnotes.\footnote{Here is a foot note. One thing that is important to me is that the footnote extends into the margin. This way there is ample room for references.}.

\paragraph{Figures} You can place entire figure floats in the main text region or in the margin. Fig.~\ref{fig:figure:example:golden} is a good example.

All we did was take \texttt{figure}$\rightarrow$\texttt{marginfigure}. You can do the same thing with \texttt{margintable}. On the other hand, sometimes you want a figure that spans the entire text width. Or, perhaps, you want several figures next to each other for comparison. To do this, we simply use the \texttt{figure*} environment. We demonstrate this in Fig.~\ref{fig:subfigure:example:lec}.
\begin{figure*}%[th]
    \centering
    \begin{subfigure}{0.3\linewidth}
    \centering
        \includegraphics[width=\linewidth]{example-image-a}
        \caption{First subfig}
        \label{fig:subfig:1:lec}
    \end{subfigure}\;%
    \begin{subfigure}{0.3\linewidth}
    \centering
        \includegraphics[width=\linewidth]{example-image-a}
        \caption{Second subfig}
        \label{fig:subfig:2:lec}
    \end{subfigure}\;%
    \begin{subfigure}{0.3\linewidth}
    \centering
        \includegraphics[width=\linewidth]{example-image-a}
        \caption{Third subfig}
        \label{fig:subfig:3:lec}
    \end{subfigure}%
    % \captionsetup{font={footnotesize,sf}}
    \caption{Here's how to spread a figure across the entire page, not just the main text width.}
    \label{fig:subfigure:example:lec}
\end{figure*}


We can place large figures in the main text and then place the figure caption in the margin notes. Compare the subfigure in Fig.~\ref{fig:subfig:3:lec} of Fig.~\ref{fig:subfigure:example:lec} to Fig.~\ref{fig:figure:example:golden:sidecap}.
\begin{figure}%[th]
    % \centering
    \sidecaption[][-2\baselineskip]{%
        Example of a margin figure. Note that the \texttt{label} command must be inside the \texttt{sidecaption}. (See source.)
        %
        %% \label command inside the \sidecaption command
        \label{fig:figure:example:golden:sidecap}
    }
    \includegraphics[width=\textwidth]{example-image-golden}
\end{figure}


\section{Other types of marginalia}

Sometimes you can use \texttt{marginnote} to place a note without a marking.%\marginnote{See?} 
\sidenote{here's another}
Note that \texttt{marginnote} does not, by default, use the same font as \texttt{sidenote}, so we had to set this in the preamble (\texttt{FlipLectureMacros.tex}). In \texttt{FlipLectureMacros} I define \verb!\sidenotenomark! which lets me write sidenotes with no mark.\sidenotenomark{Like this one!}


Maybe we want to put another type of float in the margin?\sidenote{Unfortunately, no float collision for margintables!}
What about a table, like Table~\ref{tab:margin:Table}?
% 
\begin{margintable}[-1em]
\small
    \begin{tabular}{ @{} llll @{} } \toprule % @{} removes space
        Element 
        & Core
        & Mantle
        % & $C_\text{cap}^N (\text{s}^{-1})$ 
        \\ \hline
        Iron 
        & 0.855 
        & 0.0626 
        % & $9.43\times 10^{7}$ 
        \\
        Nickel 
        & 0.052 
        & 0.00196 
        % & $7.10\times 10^{6}$ 
        \\
        Silicon 
        & 0.06 
        & 0.210 
        % & $2.24\times 10^{6}$ 
        \\
        Magnesium 
        & 0 
        & 0.228 
        % & $1.05\times 10^{6}$ 
        \\ \bottomrule
    \end{tabular}
    \captionsetup{font={scriptsize,sf}}
    \caption{Example of a margin table.}
    \label{tab:margin:Table}
\end{margintable}
One curious thing is that \texttt{margintable} does not float independently like a \texttt{sidenote}. Just be a bit careful using this since sometimes it requires manual spacing.
Another curiosity is that in my setup \texttt{sidefigure} and \texttt{sidetable} from the \texttt{sidenotes} package do not seem to work. I think may be because I played around a bit to try to make the \texttt{sidenote} font uniform.

\section{Breaking the Margin}


We define an environment \texttt{wide}\index{wide} that allows text, like equations, to spill into the margins. For example:
\begin{wide}
\begin{align}
f &= \sin\mleft(\frac{x^2}{2}\mright)
\times \arctan t 
\times \log \mleft(\cos \theta\mright)
\times \int_a^b \D{}x \exp\mleft(a^1 + b^2 + x^2\mright)
\times e^{-i\pi} 
\times \Gamma(n) 
\times _{n\!}\text{C}_m
\end{align}
\end{wide}
\begin{wide}
The definition of the margin spillover in \texttt{wide} needs to be matched to the size of the margin defined with the \texttt{geometry} package. Here's what normal text looks like.  The user must be responsible not to place any sidenotes while inside the \texttt{wide} environment.
\end{wide}


\section{Subsequent side notes}

One reason we use \texttt{sidenotes} instead of \texttt{marginnotes} is that \texttt{sidenotes} treats the notes as floating environments that do not overlap with one another.\sidenote{Here is a side note.} These side notes may cause warnings, but should not overlap.\sidenote{Here is another side note that should not overlap.}

Here's a sentence with some citations.\sidenote{Let's make this work. $\e^{i\pi} = -1$.}



\section{Some common environments}



\begin{theorem}[Euler's Identity]
\label{thm:euler:identity}
    Euler's identity\index{Euler's identity} is
    \begin{align}
        \e^{i\pi} = -1 \ .
    \end{align}
    We use the macro \verb!\e! for an upright $\e$ rather than an italicized $e$.
\end{theorem}

\begin{exercise}
\label{ex:derive:euler:identity}
    Derive Euler's identity, Thm.~\ref{thm:euler:identity}.

    Test of parskip.
\end{exercise}

\noindent Good students\index{good students} do the exercises, like Exercise~\ref{ex:derive:euler:identity}. Good instructors provide lots of examples, like Example~\ref{eg:easy:example}.

\begin{example}
\label{eg:easy:example}
    Consider the geometric series
    \begin{align}
        S = \sum_{n=0} a^n \ .
    \end{align}
    We can find a closed form expression for $S$ using
    \begin{align}
        S - aS &= 1\\
        S &= \frac{1}{1-a} \ .
    \end{align}
\end{example}

\begin{example}[named]
    test
\end{example}
\begin{bigidea}
    Why are Key Ideas and Theorems larger font? Where did I define examples to have smaller fonts?
\end{bigidea}
\begin{bigidea}[named]
    test
\end{bigidea}


\section{Some specialized environments}


\begin{bigidea}[Principle of Easy Examples]
\label{idea:easy:examples}
The examples in a book are typically much simpler than the exercises.\sidenotemark
\end{bigidea}\sidenotetext[][-2.4em]{Don't you hate it when this happens?\label{sidenote:in:environment}}

\noindent  Notice that \bigidearef{}~\ref{idea:easy:examples} has a sidenote. Ordinarily \verb!\sidnote{}! does not work in environments. To hack this manually\sidenote{And only do this sparingly!} one may use \verb!\sidenotemark! inside the environment to place the marking and then 
\begin{quotation}
\verb!\sidenotetext[][-2cm]{sidenote}!
\end{quotation}
to implement the sidenote with a manual vertical adjustment.



\section{References}

We use \texttt{biblatex} (not \BibTeX{}) to place references as footnotes.\sidenote{Unlike \BibTeX{}, biblatex does not have a fancy logo. It is simple to make one up: \BibLaTeX{}.} This is because in pedagogical material, you \emph{want} readers to engage with references. Thus it makes sense to put the reference on the same page that you refer to them rather than sequestered at the end of a chapter or---worse---the end of the document. Here is a test citation using \texttt{autocite}: some paper.\autocite{Feng:2016ijc} What is nice is that \BibLaTeX{} is clever with repeated citations.\autocite{Feng:2016ijc} Notice how it does not dump all of the bibliographic data, just what you need to remember the paper and a hyperlink to the original footnote with the full reference. It even takes \texttt{arXiv} identifies with no additional modification.


\begin{flipcomment}
Is \BibLaTeX{} suddenly not working, even thought it was working earlier? Have you been coding in a \texttt{conda} environment? \LaTeX{} may have gotten confused.
\end{flipcomment}