%% THIS IS DEPRECIATED vs. sample_fancylecturenotes
%% LaTeX Paper Template, Flip Tanedo (flip.tanedo@ucr.edu)
%% GitHub: https://github.com/fliptanedo/paper-template-2022

\documentclass[12pt, oneside]{report}    %% Has chapters

%% Load first:
%!TEX root = ../sample_paper.tex
%% Macros for lecture note typesettingj
%% Needs to be loaded before FlipPreamble.tex
%% LOADS: Fira Sans
%%  Note: when using XeLaTeX, FiraSans package loads fontspec
%%  Leads to different behavior

%%%%%%%%%%%%%%%%%%%%%%%%%%%%%%%%%%%%%
%% BibLaTeX for footnote citations %%
%%%%%%%%%%%%%%%%%%%%%%%%%%%%%%%%%%%%%

%% BibLaTeX does not want the cite package
%% from: https://tex.stackexchange.com/a/39418/8094
\makeatletter
\newcommand{\disablepackage}[2]{%
  \disable@package@load{#1}{#2}%
}
\newcommand{\reenablepackage}[1]{%
  \reenable@package@load{#1}%
}
\makeatother

%% The following line prevents cite from being loaded
\disablepackage{cite}{}

%% We use biblatex for footnote citations
\usepackage[utf8]{inputenc}     % inspire bibs, load before biblatex
\disablepackage{inputenc}{}		% disable in preamble (double loading)
\usepackage[style=verbose]{biblatex}
% In main tex file
% \addbibresource{FlipBib.bib}


%%%%%%%%%%%%%%%%%%%%%%%%%%%%%%%%%
%% Package for making an index %%
%%%%%%%%%%%%%%%%%%%%%%%%%%%%%%%%%

\usepackage{makeidx}		% For index
\makeindex

%% Use \printindex command


%%%%%%%%%%%%%%%%%%%%%%%%%%%%%%%%%%%%%
%% SIDE NOTES AND RELATED COMMANDS %%
%%%%%%%%%%%%%%%%%%%%%%%%%%%%%%%%%%%%%

\usepackage{sidenotes}  
\renewcommand*\thesidenote{\alph{sidenote}}  %% sidenotes indexed letters

% Reset sidenote numbering
\let\oldchapter\chapter
\def\chapter{%
  \setcounter{sidenote}{1}%
  \oldchapter
}


%% Sidenote font and size
%% PART I: https://tex.stackexchange.com/a/536083/8094
%% n.b. a/532251/8094 broke the sidenote floating
    \usepackage{xparse}
    \let\oldmarginpar\marginpar
    \RenewDocumentCommand{\marginpar}{om}{%
      \IfNoValueTF{#1}
        {\oldmarginpar{\mymparsetup #2}}
        {\oldmarginpar[\mymparsetup #1]{\mymparsetup #2}}}

    \newcommand{\mymparsetup}{\scriptsize\sffamily} % New, using xparse

    %% Old answer in a/532251/8094 makes all sidenotes marginnotes
    %% New answer (above) uses marginpar 
    % \renewcommand*{\marginfont}{\sffamily}    % Old
    % \renewcommand{\sidecaption}{\scriptsize\sffamily}

%% For marginnote font:
%% https://tex.stackexchange.com/questions/532245/how-to-modify-fonts-in-sidenotes/536083#536083
%% and see sidenotes documentation:
%% marginnote is a way to place notes with no mark
\renewcommand*{\marginfont}{\scriptsize\sffamily} 

\newcommand{\sidenotenomark}[1]{\sidenote[\phantom{.}]{\hspace*{-.5em}#1}}

%%%%%%%%%%%%
%%  FONT  %%
%%%%%%%%%%%%

% https://tex.stackexchange.com/a/458233/8094

%% Sans Serif Font Option for Sidenote
%% We use Sans Serif for captions and side notes. 
\usepackage[thin, scaled=1]{FiraSans} 
\DeclareCaptionStyle{sidecaption}{font={sf,footnotesize}}
\DeclareCaptionStyle{widefigure}{font={footnotesize,sf}}






%%%%%%%%%%%%%%%%%%%%%%%%%%%%%%%%%%
%% SILENCING Marginpar WARNINGS %%
%%%%%%%%%%%%%%%%%%%%%%%%%%%%%%%%%%
%% https://www.lucasshen.com/notes/tex-warnings/tex-warnings

\usepackage{silence}
\WarningFilter{latex}{Marginpar on page}
%% Silences: LaTeX Warning: Marginpar on page 1 moved.



%%%%%%%%%%%%%%%%%%%%%%%%%%%%%%%%%%
%% FORMATTING THE CHAPTER HEADER %%
%%%%%%%%%%%%%%%%%%%%%%%%%%%%%%%%%%
\usepackage{titlesec}
\titleformat{\chapter}[display]
  {\normalfont\sffamily\huge\bfseries\color{gray}}
  {\chaptertitlename\ \thechapter}{20pt}{\Huge\color{black}\textrm}
% \titleformat{\section}
%   {\normalfont\sffamily\Large\bfseries\color{cyan}}
%   {\thesection}{1em}{}




%%%%%%%%%%%%%%%%%%%%%%%%%%%%%%
%% FORMATTING THE PART PAGE %%
%%%%%%%%%%%%%%%%%%%%%%%%%%%%%%
% https://tex.stackexchange.com/a/202324
% Descriptive text after a Part is on the same page

\usepackage{xpatch}
\makeatletter
\xpatchcmd{\@endpart}{\vfil\newpage}{ \vspace{1em} }{}{}
\xpatchcmd{\@endpart}{\newpage}{}{}{}
\makeatother

       

%!TEX root = ../sample_paper.tex
%% FLIP’S PREAMBLE
%% Use FlipAdditionalHeader for project-specific packages & macros
%% Leave this general; should not need modification

%%%%%%%%%%%%%%%%%%%%%%%%%%
%%%  COMMON PACKAGES  %%%%
%%%%%%%%%%%%%%%%%%%%%%%%%%

\usepackage{amsmath}            % AMS Macros
\usepackage{amssymb}            %
\usepackage{amsfonts}           %
\usepackage{amsthm}             % 

\usepackage{graphicx}           % includegraphics
\usepackage[utf8]{inputenc}     % inspire bibs
\usepackage{./FlipTemplate/aas_macros}	% ADS bibs
\usepackage{bm}                 % \boldsymbol
\usepackage{microtype}          % improved typogarphy
\usepackage{etoolbox}           % LaTeX primitives
\usepackage[T1]{fontenc}        % CM-Super fonts

%%%%%%%%%%%%%%%%%%%%%%%%%%%
%%%  UNUSUAL PACKAGES  %%%%
%%%%%%%%%%%%%%%%%%%%%%%%%%%

%% MATH AND PHYSICS SYMBOLS
%% ------------------------
\usepackage{slashed}				% \slashed{k}
\usepackage{mathrsfs}				% Weinberg-esque letters
\usepackage{bbm}					  % \mathbbm{1} conflict: XeLaTeX 
\usepackage{cancel}					% cross out
\usepackage[normalem]{ulem} % for \sout
\usepackage{youngtab}	    	% Young Tableaux
\usepackage{mleftright}     % \mleft, \mright; bracket size/spacing
\usepackage{nicefrac}       % frac in subscript

%% CONTENT FORMAT AND DESIGN
%% -------------------------
\usepackage[dvipsnames]{xcolor}
\usepackage[hang,flushmargin]{footmisc} % no footnote indent

\usepackage{fancyhdr}		% preprint number
\usepackage{lipsum}			% block of text 
% \usepackage{tcolorbox}  % replace framed and mdframed
\usepackage[most]{tcolorbox} % `most' needed for listings
\usepackage{subcaption}	% subfigures
\usepackage{cite}			  % group cites
\usepackage{wrapfig}    

%% TABLES IN LaTeX
%% ---------------
\usepackage{booktabs}		% professional tables
\usepackage{nicefrac}		% fractions in tables,
\usepackage{multirow}		% multirow elements in a table
\usepackage{arydshln}		% dashed lines in arrays

%% Additional mods in FlipAdditionalHeader.tex

%% Other Packages and Notes
%% ------------------------
\usepackage[font=small]{caption} 	% caption font is small
\usepackage{float}         			  % for strict placement e.g. [H]
\usepackage{lineno}               % Line numbers (put \linenumbers in main text)
\usepackage{ccicons}              % Creative Commons License Icons

%%%%%%%%%%%%%%%%%%%%%%%%%%%%%%
%%%  DOCUMENT PROPERTIES  %%%%
%%%%%%%%%%%%%%%%%%%%%%%%%%%%%%

\usepackage[margin=2.5cm]{geometry} % margins; can be overwritten
\usepackage{changepage}             % overwrite geometry (e.g. lecturenotes)
\numberwithin{equation}{section}    % set equation numbering
\usepackage{marginnote}             % for \marginnote{comment}
% \usepackage{mparhack}               % fix for \marginnote
% \usepackage{marginfix}              % fix for \marginnote
% \usepackage{adjustbox}              % to rescale elements




%%%%%%%%%%%%%%%%%%%%%
%%%  TITLE DATA  %%%%
%%%%%%%%%%%%%%%%%%%%%

%% COMMANDS FOR TOP-MATTER
%% -----------------------
\newcommand{\email}[1]{\href{mailto:#1}{#1}}


%% PREPRINT NUMBER USING fancyhdr
%% Don't forget to set \thispagestyle{firststyle}
%% ----------------------------------------------
\renewcommand{\headrulewidth}{0pt}  % no separator
\setlength{\headheight}{15pt}     % min to avoid fancyhdr warning
\fancypagestyle{firststyle}{
  \rhead{\footnotesize%
  \texttt{\FlipTR}%
  }}

%% TOC overwrites fancyhdr, here's a fix
%% http://tex.stackexchange.com/questions/167828/
\usepackage{etoc}
\renewcommand{\etocaftertitlehook}{\pagestyle{plain}}
\renewcommand{\etocaftertochook}{\thispagestyle{firststyle}}



%%%%%%%%%%%%%%%%%%%%%%%%%%%
%%%  (RE)NEW COMMANDS  %%%%
%%%%%%%%%%%%%%%%%%%%%%%%%%%

%% COMMANDS FOR LATEXDIFF
%% ----------------------
%% see http://bit.ly/1M74uwc
\providecommand{\DIFadd}[1]{{\protect\color{blue}#1}} %DIF PREAMBLE
\providecommand{\DIFdel}[1]{{\protect\color{red}\protect\scriptsize{#1}}}

%% REMARK: use latexdiff option --allow-spaces
%% for \frac, ref: http://bit.ly/1iFlujR
%% Errors with environments? https://tex.stackexchange.com/q/73224

%% USAGE: latexdiff draft.tex revision.tex > diff.tex



%!TEX root = ../sample_paper.tex
%% FLIP’S MACROS 
%% USES: FlipPreamble.tex

%% FOR `NOT SHOUTING' CAPS (e.g. acronyms)
%% ---------------------------------------
\usepackage{anyfontsize} % removes scalefnt warnings
\usepackage{scalefnt}       
\newcommand\acro[1]{{\scalefont{.95}{#1}}} 


%% COMMON PHYSICS MACROS
%% ---------------------
\renewcommand{\tilde}{\widetilde}         % tilde over characters
\renewcommand{\text}{\textnormal}	        % text in equations 
\renewcommand{\vec}[1]{\mathbf{#1}}       % vectors: boldface
\newcommand{\bas}[1]{\hat{\mathbf{#1}}}   % basis vectors: hat
\newcommand{\RR}{\mathbbm{R}}
\newcommand{\CC}{\mathbbm{C}}
\renewcommand{\Im}{\operatorname{Im}}
\renewcommand{\Re}{\operatorname{Re}}

\newcommand{\abs}[1]{\left\lvert#1\right\rvert}




%% Particle Physics
%% -----------------
\newcommand\GeV{\acro{GeV}} 

%% SO(N), etc. 
%% ------------
\newcommand{\SO}[1]{\ifmmode
  \textnormal{\acro{SO(}}#1\textnormal{\acro{)}}
  \else \acro{SO($#1$)} \fi}

\newcommand{\SU}[1]{\ifmmode
  \textnormal{\acro{SU(}}#1\textnormal{\acro{)}}
  \else \acro{SU($#1$)} \fi}

\newcommand{\Sp}[1]{\ifmmode
  \textnormal{\acro{Sp(}}#1\textnormal{\acro{)}}
  \else \acro{Sp($#1$)} \fi}



%% LINEAR ALGEBRA
%% ---------------
\newcommand{\row}[1]{\tilde{\mathbf{#1}}} % row vectors have tilde

% Have to compile from CTAN ("latex undertilde.ins")
\usepackage{FlipTemplate/undertilde} 
\renewcommand{\row}[1]{\utilde{\mathbf{#1}}}   % row vectors 
\newcommand{\rbas}[1]{\row{#1}}                % row basis

\newcommand{\ket}[1]{\left|#1\right\rangle}       % <#1|
\newcommand{\bra}[1]{\left\langle#1\right|}       % |#1>

\newcommand{\aij}[2]{^{#1}_{\phantom{#1}#2}}
\newcommand{\mat}[3]{#1\aij{#2}{#3}}
\newcommand{\inv}{^{-1}}

\newcommand{\one}{\mathbbm{1}}
\newcommand{\Tr}{\operatorname{\text{Tr}}}
\newcommand*{\trans}{{\mkern-1.5mu\mathsf{T}}}     % transpose
\newcommand{\slot}{\Vtextvisiblespace[1em]{}}   % underscore to fill in



%% DIFFERENTIALS
%% -------------
%% Discouraged:
\newcommand{\dbar}{d\mkern-6mu\mathchar'26\hspace{-.1em}}    

%% Best practice: Roman differential with optional power
\newcommand{\D}[2][]{\ensuremath{\operatorname{d}\mkern-3mu^{#1}\mkern-1mu{#2}}\,}
\newcommand{\Dbar}[2][]{\operatorname{d}\mkern-10mu\mathchar'26\mkern-1mu^{#1}\mkern-1mu{#2}\,} 
%% USAGE: \Dbar[4]{\ell} 



%% TYPOGRAPHY: Best Practices
%% --------------------------
%% base of natural log is Roman
\newcommand{\e}{\operatorname{e}}
\newcommand{\E}{\operatorname{e}}  

%% imaginary number is Roman too!?
\newcommand{\I}{\operatorname{i}\mkern-2mu}  

%% phantom + for spacing (aligning in math environment)
\newcommand{\pp}{\phantom{+}}                     

%% Subscript parallel is same size as subscript perp
\usepackage{scalerel} % https://tex.stackexchange.com/a/523873/8094
\newcommand*{\paral}{{\stretchrel*{\parallel}{\perp}}}

%% For := with the dots and lines aligned, same size
%%% h/t tex.stackexchange.com/a/4881/8094
% \newcommand*{\defeq}{\mathrel{\vcenter{\baselineskip0.5ex \lineskiplimit0pt
%                      \hbox{\scriptsize.}\hbox{\scriptsize.}}}%
%                      =}
\newcommand*{\defeq}{%
  \mathrel{%
    \vcenter{%
      \baselineskip0.5ex %
      \lineskiplimit0pt %
      \hbox{\scriptsize.}\hbox{\scriptsize.}} } =}


%% Make my life easer
%% ------------------
\newcommand{\la}{\langle}
\newcommand{\ra}{\rangle}
\newcommand*{\smallslot}{\,\underline{\makebox[0.80em]{\ensuremath{}}}\,}
%   e.g. writing dual vector as <_,v> 


%% MISCELLANEOUS
%% -------------
\usepackage{pifont}
  \newcommand{\cmark}{\ding{51}}%
  \newcommand{\xmark}{\ding{55}}%

%% extension of \textvisiblespace
%% underscored space with little tick marks
%% usage: \Vtextvisiblespace[1em]
%% from: https://tex.stackexchange.com/a/50807/8094
\newcommand\Vtextvisiblespace[1][.3em]{%
  \mbox{\kern.06em\vrule height.3ex}%
  \vbox{\hrule width#1}%
  \hbox{\vrule height.3ex}}


%% FIGURES INLINE WITH EQUATIONS (e.g. Feynman diagrams)
%% -----------------------------
% \newcommand{\eqfig}[2]{%
%   \vcenter{\hbox{\includegraphics[#2]{{#1}}}}}
% %% USAGE: \eqfig{example-image-a}{width=.1\textwidth}
\newcommand{\eqfig}[1]{%
  \vcenter{\hbox{#1}}}
%% USAGE: \eqfig{\includegraphics[width=.3\textwidh]{example-image-a}}
%!TEX root = ../sample_paper.tex
%% FLIP’S MACROS FOR COMMENTS
%% USES: FlipPreamble.tex
%% These are for communicating between collaborators



%% INLINE COMMENTS
%% ---------------
%% \comment is now multiply defined with \usepackage[most]{tcolorbox}
%% \newcommand{\comment}[2]{\textcolor{red}{[\textbf{#1}: #2]}}
\newcommand{\acomment}[2]{\textcolor{red}{[\textbf{#1}: #2]}}

%% SHORT COMMENT: copy this to make your own inline comment
\newcommand{\flip}[1]{{
  \color{green!50!black}
  \footnotesize
  [\textbf{\textsf{Flip}}: \textsf{#1}]
  }}


%% USAGE EXAMPLE
%% -------------
%% \acomment{Check}{Is this equation correcT?}
%% \flip{I think it is correct.}



%% LONG, BOXED COMMENTS
%% --------------------
% Uses: tcolorbox

%% LONG COMMENT: copy this to make your own boxed comment
\newenvironment{flipcomment}
  {
    \begin{tcolorbox}[
      title=Flip Comment,
      fonttitle=\bfseries\sffamily,
      colframe=green!50!black,
      colback=white
      ]
    \small
  }{
    \end{tcolorbox}
  }

%% LONG COMMENT: copy this to make your own boxed comment
\newenvironment{boxedcomment}[1]
  {
    \begin{tcolorbox}[
      title=#1,
      fonttitle=\bfseries\sffamily,
      colframe=green!50!black,
      colback=white
      ]
    \small
  }{
    \end{tcolorbox}
  }


%% USAGE EXAMPLE:
%% --------------
%% \begin{flipcomment}
%% This is a more involved comment, perhaps with equations
%% \end{flipcomment}
%%



%% ADDING AND REMOVING TEXT
%% ------------------------
%% Analogous to LaTeXdiff-by-hand

\newcommand{\new}[1]{{ 
    \color{green!50!black}\footnotesize
    [\textbf{\textsf{New}}: {#1}]}}

%% REMOVE Environment
%% https://tex.stackexchange.com/a/488582/8094
%% Creates a nolabel environment that strips all labels
%% This is useful to avoid multiple label definitions
%% When marking old versions for deletion
\usepackage{xparse}
\ExplSyntaxOn
\NewDocumentEnvironment{nolabel}{}{
  \cs_set_eq:NN \label \use_none:n
  \cs_set_eq:cN { ltx@label} \use_none:n
}{}
\ExplSyntaxOff 

\newcommand{\remove}[1]{{
  \begin{nolabel} 
    \color{red!50!black}\footnotesize
    [\textbf{\textsf{Removed}}: {#1}]
    \end{nolabel}
    }}

%% USAGE EXAMPLE
%% --------------
%% \comment{Flip}{Can we discuss adding this:}
%% \new{This text is new compared to the last version}
%%
%% \comment{Flip}{Can we discuss removing this?}
%% \remove{Previous text that I propose removing.} 

%!TEX root = ../sample_lecturenotes.tex
%% MACROS FOR `theorem' ENVIRONMENTS IN LEC. NOTES
%% USES: FlipPreamble.tex 

%%%%%%%%%%%%%%%%%%%%%%%%%%
%% SUB-APPENDICES       %%
%%%%%%%%%%%%%%%%%%%%%%%%%%

\usepackage{appendix}   % for sub-appendices
%% see https://tex.stackexchange.com/a/120723/8094
%% ... and discussion therein


%%%%%%%%%%%%%%%%%%%%%%%%%%
%%                      %%
%% THEOREM ENVIRONMENTS %%
%% -------------------- %%
%% Provides:            %%
%%  theorem             %%
%%  exercise            %%
%%  bigidea             %%
%%                      %%
%%%%%%%%%%%%%%%%%%%%%%%%%%

%% thm title discussion
%% titles: tex.stackexchange.com/a/185168/8094
%% style: tex.stackexchange.com/a/38264/8094
%% Note: attempts to use newtcbtheorem failed
%%  no longer uses \label
%%  optional descriptions are clunky
%% Note: descriptions render in different font
%%  if you compile with pdflatex vs xelatex
%%  I can't figure out why, but leave it with pdflatex
%%  for now for maximum compatibility.

% \DeclareRobustCommand{\descriptionfira}{\fontspec{Fira Sans}}

%% DEFINE DIFFERENT STYLES (DIFF COLOR TITLES)
\newtheoremstyle{flip}% <name>
{0pt}% Space above, overwritten by tcolorbox
{0pt}% Space below, overwritten by tcolorbox
{\setlength{\parskip}{.5\baselineskip}}% <Body font>
{}% <Indent amount>
{\bfseries\sffamily\color{black}}% <Theorem head font>
{.\;}% <Punctuation after theorem head>
{.5em}% <Space after theorem headi> overwritten by tcolorbox
{}% <Theorem head spec (can be left empty, meaning `normal')>

\newtheoremstyle{flipred}% <name>
{0pt}% Space above, overwritten by tcolorbox
{0pt}% Space below, overwritten by tcolorbox
{\setlength{\parskip}{.5\baselineskip}}% <Body font>
{}% <Indent amount>
{\bfseries\sffamily\color{red!50!black}}% <Theorem head font>
{.\;}% <Punctuation after theorem head>
{.5em}% <Space after theorem headi> overwritten by tcolorbox
{}% <Theorem head spec (can be left empty, meaning `normal')>

\newtheoremstyle{flipgreen}% <name>
{0pt}% Space above, overwritten by tcolorbox
{0pt}% Space below, overwritten by tcolorbox
{\setlength{\parskip}{.5\baselineskip}}% <Body font>
{}% <Indent amount>
{\bfseries\sffamily\color{green!50!black}}% <Theorem head font>
{.\;}% <Punctuation after theorem head>
{.5em}% <Space after theorem headi> overwritten by tcolorbox
{}% <Theorem head spec (can be left empty, meaning `normal')>

\newtheoremstyle{flipblue}% <name>
{0pt}% Space above, overwritten by tcolorbox
{0pt}% Space below, overwritten by tcolorbox
{\setlength{\parskip}{.5\baselineskip}}% <Body font>
{}% <Indent amount>
{\bfseries\sffamily\color{blue!50!black}}% <Theorem head font>
{\mdseries\\}% <Punctuation after theorem head>
{.5em}% <Space after theorem headi> overwritten by tcolorbox
{\thmname{#1}\thmnumber{ #2}\thmnote{{\sffamily\firabook\quad #3}}}% <Theorem head spec (can be left empty, meaning `normal')>
% {}% <Theorem head spec (can be left empty, meaning `normal')>

%% OTHER OPTIONS
%% But renders differently in xelatex vs pdflatex
% {\thmname{#1}\thmnumber{ #2}\thmnote{{\firabook\quad #3}}}% 


\theoremstyle{flip}
\newtheorem{theorem}{Theorem}[section]

\theoremstyle{flipred}
\newtheorem{exercise}{Exercise}[section]

\theoremstyle{flipgreen}
\newtheorem{example}{Example}[section]

\theoremstyle{flipblue}
\newtheorem{bigidea}{Key Idea}[section]
    \newcommand{\bigidearef}{Key~Idea}
    \newcommand{\bigidearefs}{Key~Ideas}



% Change size of font in theorem environments
\AtBeginEnvironment{example}{\footnotesize}
\AtBeginEnvironment{exercise}{\footnotesize}
\AtBeginEnvironment{theorem}{\small}
\AtBeginEnvironment{bigidea}{\small}


% TCOLORBOX settings
% https://tex.stackexchange.com/a/633497/8094
\tcolorboxenvironment{theorem}{
    enhanced, % Skin Family `enhanced'
    colback=black!5,
    % blanker,
    frame hidden,  % no frame
    sharp corners,
    boxrule=0pt, % no gap
    % interior hidden, % no interior color
    breakable, % allows box to flow across pages
    borderline west={2pt}{0pt}{gray},
    top=10pt,
    bottom=10pt,
    before skip=20pt,
    after skip=20pt,
    separator sign none,
    separator sign= {\;}
    }

\tcolorboxenvironment{exercise}{
    enhanced, % Skin Family `enhanced'
    colback=red!5,
    borderline west={2pt}{0pt}{red!50!black},
    % blanker,
    frame hidden,  % no frame
    sharp corners,
    boxrule=0pt, % no gap
    % interior hidden, % no interior color
    breakable, % allows box to flow across pages
    top=10pt,
    bottom=10pt,
    before skip=20pt,
    after skip=20pt,
    separator sign none,
    separator sign= {\;}
    }

\tcolorboxenvironment{example}{
    enhanced, % Skin Family `enhanced'
    colback=green!5,
    borderline west={2pt}{0pt}{green!50!black},
    % blanker,
    frame hidden,  % no frame
    sharp corners,
    boxrule=0pt, % no gap
    % interior hidden, % no interior color
    breakable, % allows box to flow across pages
    top=10pt,
    bottom=10pt,
    before skip=20pt,
    after skip=20pt,
    separator sign none,
    separator sign= {\;}
    }


\tcolorboxenvironment{bigidea}{
    enhanced, % Skin Family `enhanced'
    colback=blue!5,
    borderline west={2pt}{0pt}{blue!50!black},
    % blanker,
    frame hidden,  % no frame
    sharp corners,
    boxrule=0pt, % no gap
    % interior hidden, % no interior color
    breakable, % allows box to flow across pages
    top=10pt,
    bottom=10pt,
    before skip=20pt,
    after skip=20pt,
    separator sign none,
    separator sign= {\;}
    }


%!TEX root = ../sample_paper.tex
%% Calls listings package for in-line code blocks


%% LISTINGS PACKAGE
%% https://www.overleaf.com/learn/latex/Code_listing
%% https://tex.stackexchange.com/a/350242
\usepackage{xcolor}
% \usepackage[most]{tcolorbox} %% moved to preamble
\usepackage{listings}

\definecolor{white}{rgb}{1,1,1}
\definecolor{mygreen}{rgb}{0,0.4,0}
\definecolor{light_gray}{rgb}{0.97,0.97,0.97}
\definecolor{mykey}{rgb}{0.117,0.403,0.713}

\tcbuselibrary{listings}
\newlength\inwd
\setlength\inwd{1.3cm}


% LATEX 
% https://tex.stackexchange.com/a/637305/8094
\lstdefinestyle{latexstyle}
{
  language=[LaTeX]{TeX},
  texcsstyle=*\color{blue},
  basicstyle=\ttfamily,
  moretexcs={mycommand}, % user command highlight
  frame=single,
}

%% \begin{lstlisting}[style=latexstyle]




%% PYTHON
%% from: https://tex.stackexchange.com/a/350242/8094
\newcounter{ipythcntr}
\renewcommand{\theipythcntr}{\texttt{[\arabic{ipythcntr}]}}

\newtcblisting{pyin}[1][]{%
  sharp corners,
  enlarge left by=\inwd,
  width=\linewidth-\inwd,
  enhanced,
  boxrule=0pt,
  colback=light_gray,
  listing only,
  top=0pt,
  bottom=0pt,
  overlay={
    \node[
      anchor=north east,
      text width=\inwd,
      font=\footnotesize\ttfamily\color{mykey},
      inner ysep=2mm,
      inner xsep=0pt,
      outer sep=0pt
      ] 
      at (frame.north west)
      {\refstepcounter{ipythcntr}\label{#1}In \theipythcntr:};
  }
  listing engine=listing,
  listing options={
    aboveskip=1pt,
    belowskip=1pt,
    basicstyle=\footnotesize\ttfamily,
    language=Python,
    keywordstyle=\color{mykey},
    showstringspaces=false,
    stringstyle=\color{mygreen}
  },
}
\newtcblisting{pyprint}{
  sharp corners,
  enlarge left by=\inwd,
  width=\linewidth-\inwd,
  enhanced,
  boxrule=0pt,
  colback=white,
  listing only,
  top=0pt,
  bottom=0pt,
  overlay={
    \node[
      anchor=north east,
      text width=\inwd,
      font=\footnotesize\ttfamily\color{mykey},
      inner ysep=2mm,
      inner xsep=0pt,
      outer sep=0pt
      ] 
      at (frame.north west)
      {};
  }
  listing engine=listing,
  listing options={
      aboveskip=1pt,
      belowskip=1pt,
      basicstyle=\footnotesize\ttfamily,
      language=Python,
      keywordstyle=\color{mykey},
      showstringspaces=false,
      stringstyle=\color{mygreen}
    },
}
\newtcblisting{pyout}[1][\theipythcntr]{
  sharp corners,
  enlarge left by=\inwd,
  width=\linewidth-\inwd,
  enhanced,
  boxrule=0pt,
  colback=white,
  listing only,
  top=0pt,
  bottom=0pt,
  overlay={
    \node[
      anchor=north east,
      text width=\inwd,
      font=\footnotesize\ttfamily\color{mykey},
      inner ysep=2mm,
      inner xsep=0pt,
      outer sep=0pt
      ] 
      at (frame.north west)
      {\setcounter{ipythcntr}{\value{ipythcntr}}Out#1:};
  }
  listing engine=listing,
  listing options={
      aboveskip=1pt,
      belowskip=1pt,
      basicstyle=\footnotesize\ttfamily,
      language=Python,
      keywordstyle=\color{mykey},
      showstringspaces=false,
      stringstyle=\color{mygreen}
    },
}


%!TEX root = sample_paper.tex
%% PROJECT-SPECIFIC MACROS

%% ACRONYMS and MACROS
%% Place them here. Really useful ones can 
%% migrate to ./FlipTemplate/FlipMacros.tex
%% -------------------
\newcommand{\DM}{\acro{DM}}		% nb: I do not like using this


%% COLLABORATOR COMMENTS
%% Example: replace YourName with your name
\newcommand{\YourName}[1]{{	
	\color{blue!50!black}\footnotesize
	[\textbf{\textsf{YourName}}: \textsf{#1}]}}


%% CAPTION STYLING (e.g. smaller caption)
\captionsetup{font={scriptsize,sf}} %% caption package


%%%%%%%%%%%%
%% LAYOUT %%
%%%%%%%%%%%%

% \usepackage[margin=2.5cm]{geometry}

%% References in two columns, smaller (for papers)
%% http://tex.stackexchange.com/questions/20758/
%% \usepackage{etoolbox} %% called above
%% -----------------------------------------------
% \usepackage{multicol}
% \usepackage{relsize}
% \patchcmd{\thebibliography}
%   {\list}
%   {\begin{multicols}{2}\smaller\list}
%   {}
%   {}
% \appto{\endthebibliography}{\end{multicols}}
%% -----------------------------------------------

% Change list spacing (instead of package paralist)
% from: http://en.wikibooks.org/wiki/LaTeX/List_Structures#Line_spacing
% alternative: enumitem package
\let\oldenumerate\enumerate
\renewcommand{\enumerate}{
  \oldenumerate
  \setlength{\itemsep}{4pt}
  \setlength{\parskip}{0pt}
  \setlength{\parsep}{0pt}
}

\let\olditemize\itemize
\renewcommand{\itemize}{
  \olditemize
  \setlength{\itemsep}{4pt}
  \setlength{\parskip}{0pt}
  \setlength{\parsep}{0pt}
}




%% TABLES / TABULAR ENVIRONMENT
%% -----------------------------
%% Vertical spacing between rows
% \renewcommand{\arraystretch}{1.5} 

%% More spacing between table columns
% \setlength{\tabcolsep}{12pt}


%% TYPOGRAPHY
%% ----------
%% French spacing: only one space after periods
% \frenchspacing

%% NEW ENVIRONMENTS
%% ----------------
%% EXAMPLE: framed environments
% \newmdtheoremenv[
%     skipabove=2em,
%     skipbelow=2em,
%     linewidth=5pt,
%     linecolor=red!50!black,
%     topline=false,
%     rightline=false,
%     bottomline=false
%     ]{framed}{Frame}[section]




%%%%%%%%%%%%%%%%%%%%%%%%%%%%%%%%%%%
%% JUST FOR THE SAMPLE TEMPLATES %%
%%%%%%%%%%%%%%%%%%%%%%%%%%%%%%%%%%%

%% FOR THE sample_paper.tex TEMPLATE
%% Can be deleted for real projects
%% ---------------------------------
\usepackage{xspace}    % only to explain why NOT to sue this

%% Easy arXiv hyperlinks (in sample_shortnote.tex)
\newcommand{\arXiv}[1]{%
	\texttt{%
	\href{https://arxiv.org/abs/#1}{arXiv:{#1}}
	}
	}

%% RENEW COMMANDS, example
% \newcommand{\LaTeXx}{\LaTeX{}}
\def\BibTeX{{\rm B\kern-.05em{\sc i\kern-.025em b}\kern-.08em
    T\kern-.1667em\lower.7ex\hbox{E}\kern-.125emX}{}}

\def\BibLaTeX{{\rm B\kern-.05em{\sc i\kern-.025em b}\kern-.08em
    \LaTeX}{}}





    %% Modify this for each project

%% Load last:
\input{FlipTemplate/FlipPreambleEnd}

%%%%%%%%%%%%%%%%%%%%%%%%%%%%
%% LECTURE NOTES SETTINGS %%
%%%%%%%%%%%%%%%%%%%%%%%%%%%%

% \linenumbers                  %% print line numbers (lineno package)
\graphicspath{{figures/}}       %% figure folder
\addbibresource{FlipBib.bib}    %% Define BibLaTeX source(s)

%% LEAVE THESE HERE 

% \usepackage{geometry}
\geometry{                      %% large margin for side notes
    paper=letterpaper, 
    hmargin={1cm,7.25cm},       %% 6.25cm space on right
    vmargin={2cm,2cm}, 
    marginparsep=.5cm, 
    marginparwidth=5.75cm
}

%% Def. full width; uses changepage package; 6.25cm to match hmargin difference;
\newenvironment{wide}{\begin{adjustwidth}{0cm}{-6.25cm}}{\end{adjustwidth}}


% Reset the sidenote number each section 
\let\oldsection\section
\def\section{%
  \setcounter{sidenote}{1}%
  \oldsection
}




\begin{document}


\newgeometry{margin=2cm}                   % plain geometry for frontmatter
\newcommand{\FlipTR}{UCR-TR-2025-FLIP-00X} % TR#, pdfsync may fail on 1st page
\thispagestyle{firststyle} 	               % TR#; otherwise use \thispagestyle{empty}
\pagenumbering{gobble}                     % no page number on first page 

%%%%%%%%%%%%%%%%%%%%%%%%
%%%  FRONTMATTER    %%%%
%%%%%%%%%%%%%%%%%%%%%%%%


\begin{center}
    {\large \textsf{UC Riverside Physics XXX, Fall 20YY} \par}
    {\huge \textbf{Course Title} \par}\vspace{.5em}
    {\large {Descriptive subtitle} \par}
    \vskip .5cm
\end{center}

%!TEX root = paper.tex
%% Update the above with the appropriate root

%% AUTHOR LIST: separated to keep main file clean
%% This is not part of the document that is updated often.

%% Multi line institution?  Use \phantom{$^{c}$\,}

%%%%%%%%%%%%%%%
%%  AUTHORS  %%
%%%%%%%%%%%%%%%

\newcommand{\authorA}{Flip Tanedo}
\newcommand{\emailA}{flip.tanedo@ucr.edu}
\newcommand{\orcidA}{0000-0003-4642-2199}
\newcommand{\institutionA}{
	Department of Physics \& Astronomy, 
	University of  California, Riverside, 
	\normalfont{CA} 92521 \normalfont{USA}}

\newcommand{\authorB}{Your Name}
\newcommand{\emailB}{your.name@ucr.edu}
\newcommand{\orcidB}{0000-0003-4642-2200}
\newcommand{\institutionB}{
	Department of Physics \& Astronomy, 
	University of  California, Elsewhere, 
	\normalfont{CA} 99999 \normalfont{USA}}

\newcommand{\authorC}{Tu Nombre}
\newcommand{\emailC}{tu.nombre@ucr.edu}
\newcommand{\orcidC}{0000-0003-4642-2201}
\newcommand{\institutionC}{
	Department of Physics \& Astronomy
	and
	Institute of Some Long-Named Topic, 
	\\ \phantom{$^{c}$\,}
	University of New Line, Elsewhere City, 
	\normalfont{XY} 99999 \normalfont{USA}}


%%%%%%%%%%%%%%%%%%
%%  FORMATTING  %%
%%%%%%%%%%%%%%%%%%

\begin{center}
	\textbf{\authorA}$^{a}$,
	\textbf{\authorB}$^{b}$,
	and
	\textbf{\authorC}$^{c}$
	\par

	\texttt{\footnotesize \email{\emailA}}~\orcidlink{\orcidA},
	\texttt{\footnotesize \email{\emailB}}~\orcidlink{\orcidB},
	\texttt{\footnotesize \email{\emailC}}~\orcidlink{\orcidC}
\end{center}


% quotation environment is same width as abstract
\begin{quotation}\noindent
	\footnotesize
	\noindent$^{a}$
	\textit{\institutionA} 
	\\ $^{b}$ \textit{\institutionB} 
	\\ $^{c}$ \textit{\institutionC} 
\end{quotation}


\vspace{2em}\noindent
This is a good place to put the course description and any commentary about the nature of these notes. 


\vspace{2em}
\noindent
\textsf{Last Compiled: \today}

\noindent
\textsf{CC BY-NC-SA 4.0}~\ccbyncsa 

\noindent % Course notes URL
% \url{https://github.com/fliptanedo/P231-2023-Math-Methods}

%% Front page logos
\vspace*{\fill}
\begin{center}
\includegraphics[height=.1\textwidth]{figures/FlipAmbigram.png}
\hspace{5em}
\includegraphics[height=.1\textwidth]{figures/UCRPnA_banner.png}
\end{center}

\newpage

\small
\setcounter{tocdepth}{2}
\tableofcontents
\normalsize
\clearpage
\restoregeometry        %% Return to lecture note geometry 
\pagenumbering{arabic}  %% Turn on regular page numbers


%%%%%%%%%%%%%%%%%%%%%
%%%  THE CONTENT  %%%
%%%%%%%%%%%%%%%%%%%%%

\chapter{The Lecture Notes Template}

\section{Usage}

Unlike the paper template, the lecture note template is organized into chapters and has large margins for side notes. To use chapters, we call the \texttt{report} class rather than the \texttt{article} class. This template also uses \texttt{biblatex} rather than \texttt{bibtex} to compile.


\section{Things I'm working on ...}

\subsection{Sidecite}
\verb!\sidecite! does not work. Perhaps this may help?
\url{https://github.com/fmarotta/kaobook/issues/15}


\subsection{Includeonly}

I primarily use \verb!\input{}! to chunk up my source because it does not introduce a \verb!\clearpage! the way that \verb!\include{}! does. However, when using \verb!\include{}! one may use the \verb!\includeonly{}! command to compile just one part of a document while \emph{maintaining all references} to other parts of the document.\footnote{\url{https://tex.stackexchange.com/a/250/8094}} 



% BUT: REQUIRES XELATEX
%!TEX root = ../sample_lecturenotes.tex
%% Update the above with the appropriate root

\chapter{Margins and Figures} %% if using report class
% \chapterminitoc

The main difference between my lecture notes and my paper template is the large margin to make use of side notes. The margin is a natural place to encourage students to jot notes and to host useful marginalia.%
\begin{marginfigure}%[th]
    \includegraphics[width=.8\textwidth]{example-image-golden}
    \captionsetup{font={scriptsize,sf}}
    \caption{Example of a margin figure.}
    \label{fig:figure:example:golden}
\end{marginfigure}
You can even place figures in the margin, see Fig.~\ref{fig:figure:example:golden}.\sidenote{Look! No float collision!}

\section{Vision}

This document is inspired by Edward Tufte and the implementation of his ideals in the \texttt{tufte-latex} package. The notes have a large margin for side notes and floats (e.g.\ figures).  Visually this means that the column of main text is narrower, which permits a slightly smaller font.


\section{Using the Margin}

We implement marginalia with the \texttt{sidenotes} package\index{side notes}. We highlight the main usage here. A standard sidenote uses \verb!\sidenote{...}! and looks like this\sidenote{Test of a sidenote. Let's add some extra text here to demonstrate the following point about non-overlapping notes}. Unlike \verb!marginnote!s, \verb!sidenotes! do not overlap with each other.\sidenote{An example of a sidenote that does not overlap with the previous one.} What more, the sidenotes coexist fine with footnotes.\footnote{Here is a foot note. One thing that is important to me is that the footnote extends into the margin. This way there is ample room for references.}.

\paragraph{Figures} You can place entire figure floats in the main text region or in the margin. Fig.~\ref{fig:figure:example:golden} is a good example.

All we did was take \texttt{figure}$\rightarrow$\texttt{marginfigure}. You can do the same thing with \texttt{margintable}. On the other hand, sometimes you want a figure that spans the entire text width. Or, perhaps, you want several figures next to each other for comparison. To do this, we simply use the \texttt{figure*} environment. We demonstrate this in Fig.~\ref{fig:subfigure:example:lec}.
\begin{figure*}%[th]
    \centering
    \begin{subfigure}{0.3\linewidth}
    \centering
        \includegraphics[width=\linewidth]{example-image-a}
        \caption{First subfig}
        \label{fig:subfig:1:lec}
    \end{subfigure}\;%
    \begin{subfigure}{0.3\linewidth}
    \centering
        \includegraphics[width=\linewidth]{example-image-a}
        \caption{Second subfig}
        \label{fig:subfig:2:lec}
    \end{subfigure}\;%
    \begin{subfigure}{0.3\linewidth}
    \centering
        \includegraphics[width=\linewidth]{example-image-a}
        \caption{Third subfig}
        \label{fig:subfig:3:lec}
    \end{subfigure}%
    % \captionsetup{font={footnotesize,sf}}
    \caption{Here's how to spread a figure across the entire page, not just the main text width.}
    \label{fig:subfigure:example:lec}
\end{figure*}


We can place large figures in the main text and then place the figure caption in the margin notes. Compare the subfigure in Fig.~\ref{fig:subfig:3:lec} of Fig.~\ref{fig:subfigure:example:lec} to Fig.~\ref{fig:figure:example:golden:sidecap}.
\begin{figure}%[th]
    % \centering
    \sidecaption[][-2\baselineskip]{%
        Example of a margin figure. Note that the \texttt{label} command must be inside the \texttt{sidecaption}. (See source.)
        %
        %% \label command inside the \sidecaption command
        \label{fig:figure:example:golden:sidecap}
    }
    \includegraphics[width=\textwidth]{example-image-golden}
\end{figure}


\section{Other types of marginalia}

Sometimes you can use \texttt{marginnote} to place a note without a marking.%\marginnote{See?} 
\sidenote{here's another}
Note that \texttt{marginnote} does not, by default, use the same font as \texttt{sidenote}, so we had to set this in the preamble (\texttt{FlipLectureMacros.tex}). In \texttt{FlipLectureMacros} I define \verb!\sidenotenomark! which lets me write sidenotes with no mark.\sidenotenomark{Like this one!}


Maybe we want to put another type of float in the margin?\sidenote{Unfortunately, no float collision for margintables!}
What about a table, like Table~\ref{tab:margin:Table}?
% 
\begin{margintable}[-1em]
\small
    \begin{tabular}{ @{} llll @{} } \toprule % @{} removes space
        Element 
        & Core
        & Mantle
        % & $C_\text{cap}^N (\text{s}^{-1})$ 
        \\ \hline
        Iron 
        & 0.855 
        & 0.0626 
        % & $9.43\times 10^{7}$ 
        \\
        Nickel 
        & 0.052 
        & 0.00196 
        % & $7.10\times 10^{6}$ 
        \\
        Silicon 
        & 0.06 
        & 0.210 
        % & $2.24\times 10^{6}$ 
        \\
        Magnesium 
        & 0 
        & 0.228 
        % & $1.05\times 10^{6}$ 
        \\ \bottomrule
    \end{tabular}
    \captionsetup{font={scriptsize,sf}}
    \caption{Example of a margin table.}
    \label{tab:margin:Table}
\end{margintable}
One curious thing is that \texttt{margintable} does not float independently like a \texttt{sidenote}. Just be a bit careful using this since sometimes it requires manual spacing.
Another curiosity is that in my setup \texttt{sidefigure} and \texttt{sidetable} from the \texttt{sidenotes} package do not seem to work. I think may be because I played around a bit to try to make the \texttt{sidenote} font uniform.

\section{Breaking the Margin}


We define an environment \texttt{wide}\index{wide} that allows text, like equations, to spill into the margins. For example:
\begin{wide}
\begin{align}
f &= \sin\mleft(\frac{x^2}{2}\mright)
\times \arctan t 
\times \log \mleft(\cos \theta\mright)
\times \int_a^b \D{}x \exp\mleft(a^1 + b^2 + x^2\mright)
\times e^{-i\pi} 
\times \Gamma(n) 
\times _{n\!}\text{C}_m
\end{align}
\end{wide}
\begin{wide}
The definition of the margin spillover in \texttt{wide} needs to be matched to the size of the margin defined with the \texttt{geometry} package. Here's what normal text looks like.  The user must be responsible not to place any sidenotes while inside the \texttt{wide} environment.
\end{wide}


\section{Subsequent side notes}

One reason we use \texttt{sidenotes} instead of \texttt{marginnotes} is that \texttt{sidenotes} treats the notes as floating environments that do not overlap with one another.\sidenote{Here is a side note.} These side notes may cause warnings, but should not overlap.\sidenote{Here is another side note that should not overlap.}

Here's a sentence with some citations.\sidenote{Let's make this work. $\e^{i\pi} = -1$.}



\section{Some common environments}



\begin{theorem}[Euler's Identity]
\label{thm:euler:identity}
    Euler's identity\index{Euler's identity} is
    \begin{align}
        \e^{i\pi} = -1 \ .
    \end{align}
    We use the macro \verb!\e! for an upright $\e$ rather than an italicized $e$.
\end{theorem}

\begin{exercise}
\label{ex:derive:euler:identity}
    Derive Euler's identity, Thm.~\ref{thm:euler:identity}.

    Test of parskip.
\end{exercise}

\noindent Good students\index{good students} do the exercises, like Exercise~\ref{ex:derive:euler:identity}. Good instructors provide lots of examples, like Example~\ref{eg:easy:example}.

\begin{example}
\label{eg:easy:example}
    Consider the geometric series
    \begin{align}
        S = \sum_{n=0} a^n \ .
    \end{align}
    We can find a closed form expression for $S$ using
    \begin{align}
        S - aS &= 1\\
        S &= \frac{1}{1-a} \ .
    \end{align}
\end{example}

\begin{example}[named]
    test
\end{example}
\begin{bigidea}
    Why are Key Ideas and Theorems larger font? Where did I define examples to have smaller fonts?
\end{bigidea}
\begin{bigidea}[named]
    test
\end{bigidea}


\section{Some specialized environments}


\begin{bigidea}[Principle of Easy Examples]
\label{idea:easy:examples}
The examples in a book are typically much simpler than the exercises.\sidenotemark
\end{bigidea}\sidenotetext[][-2.4em]{Don't you hate it when this happens?\label{sidenote:in:environment}}

\noindent  Notice that \bigidearef{}~\ref{idea:easy:examples} has a sidenote. Ordinarily \verb!\sidnote{}! does not work in environments. To hack this manually\sidenote{And only do this sparingly!} one may use \verb!\sidenotemark! inside the environment to place the marking and then 
\begin{quotation}
\verb!\sidenotetext[][-2cm]{sidenote}!
\end{quotation}
to implement the sidenote with a manual vertical adjustment.



\section{References}

We use \texttt{biblatex} (not \BibTeX{}) to place references as footnotes.\sidenote{Unlike \BibTeX{}, biblatex does not have a fancy logo. It is simple to make one up: \BibLaTeX{}.} This is because in pedagogical material, you \emph{want} readers to engage with references. Thus it makes sense to put the reference on the same page that you refer to them rather than sequestered at the end of a chapter or---worse---the end of the document. Here is a test citation using \texttt{autocite}: some paper.\autocite{Feng:2016ijc} What is nice is that \BibLaTeX{} is clever with repeated citations.\autocite{Feng:2016ijc} Notice how it does not dump all of the bibliographic data, just what you need to remember the paper and a hyperlink to the original footnote with the full reference. It even takes \texttt{arXiv} identifies with no additional modification.


\begin{flipcomment}
Is \BibLaTeX{} suddenly not working, even thought it was working earlier? Have you been coding in a \texttt{conda} environment? \LaTeX{} may have gotten confused.
\end{flipcomment}

% \chapter{Paper examples}

% Here are the standard examples I use for my \texttt{paper} template. I include them here to check that nothing has broken. These do not make use of the margin at all. You can see what happens when some text spills into the margin unintentionally.

% %!TEX root = ../sample_paper.tex
%% Update the above with the appropriate root
\section{Common environments}

\subsection{Figures: floating and wrapped}

\begin{figure}%[th]
    \centering
    \includegraphics[width=0.4\textwidth]{example-image-a}
    \caption[The figure environment shows up often. Here's a trick for footnotes in a floating caption.]{The figure environment shows up often. Here's a trick for footnotes in a floating caption.\footnotemark
    \label{fig:figure:example}}
\end{figure} 
\footnotetext{Use the command \texttt{footnotemark} inside the caption then the command \texttt{footnotetext} just outside the environment. This requires the caption to have an optional argument that contains the text that should show up in a list of figures.}


\begin{figure}%[th]
    \centering
	\begin{subfigure}{0.3\textwidth}
    \centering
        \includegraphics[width=\linewidth]{example-image-a}
        \caption{First subfig}
        \label{fig:subfig:1}
    \end{subfigure}\;%
    \begin{subfigure}{0.3\textwidth}
    \centering
        \includegraphics[width=\linewidth]{example-image-a}
        \caption{Second subfig}
        \label{fig:subfig:2}
    \end{subfigure}\;%
    \begin{subfigure}{0.3\textwidth}
    \centering
        \includegraphics[width=\linewidth]{example-image-a}
        \caption{Third subfig}
        \label{fig:subfig:3}
    \end{subfigure}%
    \caption{Here's how to use subfigures}
    \label{fig:subfigure:example}
\end{figure}

Use 
% \textbackslash\texttt{centering}
\verb!\centering!
rather than the \texttt{center} environment in figure environments to avoid adding extra vertical space.\footnote{\url{https://tex.stackexchange.com/a/23653/8094}\label{foot:centering}}

\begin{wrapfigure}{l}{0.3\textwidth}
	\includegraphics[width=0.9\linewidth]{example-image-a}
	\caption{via \texttt{wrapfigure}.}
	\label{fig:wrapfig}
\end{wrapfigure}
\lipsum[1]

\subsection{Figures in Equation Environments}
\label{sec:figs}
A trick with: 
% \verb!\eqfig{example-image-a}{width=.1\textwidth}!. 
% 
% \begin{quote}
% \verb!\eqfig{\includegraphics[width=.1\textwidth]{{example-image-a}}}!
% \end{quote}
% 
This is a command that we define in \texttt{FlipMacros.tex}. \flip{It may be better to define this so that users input their own includegraphics.}

\begin{align}
	% \vcenter{
	% 	\hbox{\includegraphics[width=.1\textwidth]{{example-image-a}}}
	% 	}
	\eqfig{\includegraphics[width=.1\textwidth]{{example-image-a}}}
	&=
	i g \gamma^\mu \ . 
	\label{eq:vector}
	\\
	\vcenter{
		\hbox{\includegraphics[width=.1\textwidth]{{example-image-a}}}
		}
	&=
	g \gamma^\mu\gamma^5 \ . 
	\label{eq:axial}
	\\
	\vcenter{
		\hbox{\includegraphics[width=.1\textwidth]{{example-image-a}}}
		}
	&=
	ig  \ . 
	\label{eq:scalar}
	\\
	% \vcenter{
	% 	\hbox{\includegraphics[width=.1\textwidth]{{example-image-a}}}
	% 	}
	%% Shortcut: use \eqfig command
	% \eqfig{example-image-a}{width=.1\textwidth}
	\eqfig{\includegraphics[width=.1\textwidth]{{example-image-a}}}
	&=
	g \gamma^5 \ . 
	\label{eq:pseudo}
\end{align}


\subsection{Best practices for tables}
\label{sec:tables}

% \begin{table}
	% \renewcommand{\arraystretch}{1.3} % spacing between rows
	% \centering
	\begin{tabular}{ @{} llll @{} } \toprule % @{} removes space
		Element 
		& Core MF 
		& Mantle MF 
		& $C_\text{cap}^N (\text{s}^{-1})$ 
		\\ \midrule
		Iron 
		& 0.855 
		& 0.0626 
		& $9.43\times 10^{7}$ 
		\\
		Nickel 
		& 0.052 
		& 0.00196 
		& $7.10\times 10^{6}$ 
		\\
		Silicon 
		& 0.06 
		& 0.210 
		& $2.24\times 10^{6}$ 
		\\
		Magnesium 
		& 0 
		& 0.228 
		& $1.05\times 10^{6}$ 
		\\ \bottomrule
	\end{tabular}
	% \caption{
		% Mass fractions of the Earth's core and mantle.
		% \label{table:elements}
% 	}
% \end{table}




\section{Labels and cleveref}
\label{sec:labels:and:cleveref}

\subsection{\texorpdfstring{\texttt{cleveref}}{cleveref}}
\label{sec:cleveref}

\texttt{cleveref} is a handy package when referring to ranges of equations. 

\begin{itemize}
	\item Using \texttt{amsmath.sty}'s \texttt{eqref}: \eqref{eq:pseudo}
	\item Using \texttt{cleverefs}'s \texttt{cref}: \cref{eq:pseudo}
\end{itemize}

For a range of equations:
\begin{itemize}
	\item Using \texttt{amsmath.sty}'s \texttt{eqref}: \eqref{eq:vector} -- \eqref{eq:pseudo}
	\item Using \texttt{cleverefs}'s \texttt{crefrange}: \crefrange{eq:vector}{eq:pseudo}
\end{itemize}

For several equations with the range done automatically:
\begin{itemize}
	\item Using \texttt{amsmath.sty}'s \texttt{eqref}: \eqref{eq:vector}, \eqref{eq:axial}, \eqref{eq:scalar}, \eqref{eq:pseudo}
	\item Using \texttt{cleverefs}'s \texttt{cref}: \cref{eq:vector,eq:pseudo,eq:axial,eq:scalar}
\end{itemize}

\texttt{cleveref} automatically identifies the type of object it is referring to. Thus you can use \verb!\cref! to refer to any label, for example \cref{foot:centering}.\footnote{Note that this example does not work: the automatic text does not say `footnote,' it says `section.'} You can use \verb!\Cref! to have a capitalized the cross reference name. For example: the sections above are \Cref{sec:macros,sec:cleveref,sec:figs,sec:tables}.


\subsection{Sub-equations}

One can also wrap a \texttt{align} environment with a \texttt{subequations} environment. The \texttt{subequations} environment can be given a label. For example,
\begin{subequations}\label{eq:subequations}
\begin{align}
	a &= \pi 
	\label{eq:subequation:1}
	\\
	b &= \e^{\I \pi} 
	\ .
	\label{eq:subequation:2}
\end{align}
\end{subequations}
where we can now refer to the pair of equations \eqref{eq:subequations} or simply one of the equations, \eqref{eq:subequation:2}. This also works in \texttt{cleveref}: \cref{eq:subequations} and \cref{eq:subequation:2}.


\subsection{Referring to Equations}

One style suggestion is to use parentheses to refer to an equation with no additional modifiers \emph{except} at the beginning of a sentence.\footnote{\url{https://academia.stackexchange.com/a/21793}} For example: ``The second term in (3)...'' and ``Equation~(3) has two terms...''



\section{Macros}
\label{sec:macros}


\subsection{Modest capitalization}

Small caps are useful when your text contains acronyms and you do not want them to visually imply emphasis. In other words, we can use them as `not shouting' capitalization. We define a macro \texttt{acro} for this purpose. The default is for \texttt{acro} to be a wrapper for 
\verb!\scalefont{.95}!. Here is an example:
\begin{itemize}
	\item \acro{AdS} in \acro{5D} at the \acro{LHC}.\footnote{\acro{AdS} in \acro{5D} at the \acro{LHC}} (using \texttt{acro})
	\item AdS in 5D at the LHC.\footnote{AdS in 5D at the LHC.} (no \texttt{acro})
	\item {\small{AdS}} in {\small{5D}} at the {\small{LHC}}.\footnote{{\small{AdS}} in {\small{5D}} at the {\small{LHC}}. (Note \texttt{small} is larger than \texttt{footnotesize}.)} (using \texttt{small}, footnote is messed up)
\end{itemize}
The default Computer Modern font leads to harmless warnings when using the \verb!\scalefnt! package. You can remove these by calling the \verb!anyfontsize! package. To change the font scaling, modify the following line in \verb!FlipMacros.tex!:
\begin{quote}
\verb!\newcommand\acro[1]{{\scalefont{.95}{#1}}}!
\end{quote}
Reasonable choices are between .9 and .95. It is a standard practice to use small caps for acronyms and initialisms~\cite{bringhurst2012elements}. The \emph{New York Times} uses small caps for acronyms longer than three letters, but I find that in physics we have so many three-letter acronyms that a more aggressive use of small caps for all acronyms and initialisms helps with readability.


\subsection{Macros for Collaboration}

There are many ways to add notes when collaborating on a document. I like in-line notes with an author name and a color.  \flip{This is an example of a comment.} It is also useful to have a macro for highlighting new text and for proposing the removal of old text.\footnote{Git does this automatically at the level of source code. \texttt{LaTeXDiff} ostensibly does this automatically, but is prone to compile errors and is notoriously difficult to troubleshoot.}

\new{
I fixed the equation:
\begin{align}
	a=b^2 
	\label{eq:samename}
\end{align}
It has label \texttt{eq:samename}. 
}

\remove{
	% It is good practice to indent the `to-be deleted' text
	Here is an equation:
	\begin{align}
	a=b 
	\label{eq:samename}
	\end{align}
	It has label \texttt{eq:samename}.
}

This is essentially a manual version of the \texttt{latexdiff} command. This command can be notoriously fussy around math environments. I personally advocate for using \texttt{git}-related tools to quickly identifying where a version was edited and then using tags to identify edits that need to be highlighted for further discussion. One nice thing about the \texttt{remove} tag above is that it also strips any \texttt{label}s so that there are no `multiply defined label' warnings and one can uniquely refer to a single equation, \eqref{eq:samename}.

\begin{flipcomment}
This is an extended comment that shows up as a text box. I might use this to make some ponderous point about why I think my version of a draft paragraph is more appropriate than yours.
\end{flipcomment}

\begin{boxedcomment}{Custom Title}
You can also define boxes that take customized titles. 
\end{boxedcomment}


\subsection{\texorpdfstring{\texttt{xspace}}{xspace}}

The \verb!\xspace! command is useful at the end of a macro. It stands for: ``insert a space if and only if there is supposed to be a space.'' It can be useful for spacing with user-defined macros. Consider the following examples:
\begin{itemize}
	\item Without \texttt{xspace}: \LaTeX typesetting...
	\item With \texttt{xspace}: \LaTeX\xspace typesetting...
	\item Without \texttt{xspace}: Typeset with \LaTeX.
	\item With \texttt{xspace}: Typeset with \LaTeX\xspace.
\end{itemize}
However, one should consider \verb!\xspace! depreciated.  The results can be unreliable.\footnote{\url{https://tex.stackexchange.com/a/86620/8094}}  For example, one could define \verb!\newcommand{\test}{{test}\xspace}!, note the extra pair of braces in the definition. Even though there is an \verb!\xspace!, this command will fail to place a space between \verb!\test \test!. This type of problem shows up for me because I have a macro \verb!\acro{}! for making acronyms smaller. If I define a shortcut \verb!\newcommand{\DM}{\acro{DM}}! then there is no space between \verb!\DM \test!. The result is: \acro{DM}\xspace {test}. 

The suggested practice is to end your commands with empty braces or a slash: \verb!\DM{}! or \verb!\DM\!. All spacing works out as intended. 
\begin{itemize}
	\item \verb!\DM{} Halo! produces \DM{} Halo
	\item \verb!\LaTeX Fails.! produces \LaTeX Fails.
	\item \verb!\LaTeX{} Works.! produces \LaTeX{} Works.
\end{itemize}






\section{Mathematics and Physics}

\subsection{Environments}

Use the \texttt{align} environment instead of \texttt{eqnarray}, see the TeXblog discussion\footnote{\url{https://texblog.net/latex-archive/maths/eqnarray-align-environment/}}. The double dollar sign notation for \texttt{displaymath} is depreciated.\footnote{\url{https://tex.stackexchange.com/questions/503/why-is-preferable-to}} The suggested alternative is \verb!\[! and \verb!\]!, though this is rather annoying. As a default I always use \texttt{align}.

Single dollar sign \verb!$! notation is also depreciated relative to \verb!\(! and \verb!\)! for inline text. However, I am stubborn about keeping in-line text as simple as possible. The dollar sign is a single character and it is visually easier to identify as a delimiter for math mode. I thus continue using single dollar signs for inline mathematics. 


\subsection{Text and Math: super- and subscripts}

Use the \texttt{text} command to insert text into math environments. For subscripts use \texttt{textnormal}. 
% 
We use the \texttt{textnormal} command for super- or subscripts rather than the \texttt{text} command because this automatically uses the correct size. Otherwise, subscripts might be oddly large---for example, when writing $x_\text{min}$ in a footnote, the $\text{min}$ might not know that needs to be smaller than it would be in the main text. In \texttt{FlipMacros.tex} we redefine \verb!\text! to be \verb!\textnormal! so that you can simply write \verb!\text!.
% Compare, for example: $G_\textnormal{D},\, G_\text{D}$.

One place this shows up is if you have a subscript that is not a mathematical variable. For example, $x_a$ makes sense for the value of $x$ at point $a$, but $x_\text{b}$ should be used if the `b' is shorthand for boundary. Similarly, $E_\text{max}$ for the maximum energy, rather than $E_{max}$.


\subsection{Units and spacing}

Use a tie (\verb!~!) to enforce a non-breaking short space between a number and its units: $0.5~\text{MeV}$ is written as \verb!$0.5~\text{MeV}$!. Units should not be italicized. If you want to be svelte you can use a thin space (\verb!\,!).  Some people like the \texttt{siunitx} package; I find it a little cumbersome for that it is, especially given that I usually write in natural units. The Physical Review Style Guide suggests a space (which I take to be a tie) between a number and its unit and a thin space between consecutive units:
\begin{align}
	6\times 10^{-34}~\text{m}^2\,\text{kg}/\text{s}
\end{align}
is written as \verb!6\times 10^{-34}~\text{m}^2\,\text{kg}/\text{s}!.


Some care is required for spacing with math operators\footnote{\url{https://tex.stackexchange.com/a/35585/8094}}. Here's a guideline for how to use different bits of manual spacing\footnote{\url{https://tex.stackexchange.com/questions/25810/when-one-should-use-spacing-line-quad-or}}.


\subsection{Upright characters}

These come from the \acro{ISO 80000} standards for typesetting mathematics and physics.\footnote{See discussion in \url{https://tex.stackexchange.com/q/14821/8094}} It is not obvious to me that these are applicable to the typographical culture of physics, but the most important thing is to be consistent.
\begin{itemize}
	\item Units are always upright, \textmu m is a micrometer. You can use various unit packages to do this automatically. 
	\item The base of the natural logarithm is upright $\e^{i\pi}$ versus $e^{i\pi}$. The best physics argument is to avoid confusion between the exponential $\e$ and the electric coupling $\alpha = e^2/4\pi$.\footnote{I thank Matt Reece for pointing this out to me.}
	\item The differential is an operator so it should be upright. I have macros for this: $dx$ vs.~$\D{x}$ and $\dbar p$ vs.~$\Dbar{p}$. The \texttt{physics} package has macros for this. 
	\item You could also do this for the imaginary number: $a+\I b = \e^{\I \theta}$. This one is a little trickier to get the spacing right. We use \verb!\newcommand{\I}{\operatorname{i}\mkern-2mu}!.
	\item You can use \verb!\DeclareMathOperator! to define upright Roman letters that should be treated as a single operator like $\sin$. This also automatically fixes the spacing after the operator contextually depending on whether the next character is understood as an argument, $\sin x$, or a group $\sin(ax)$.
\end{itemize}
There is a historical discussion on \texttt{hsm.stackexchange}\footnote{\url{https://hsm.stackexchange.com/questions/6727/fracdydx-versus-frac-mathrm-dy-mathrm-dx}}.  For all of these, it helps to define macros. This makes it easy to change the style when you have a co-author who strongly disagrees.

\subsection{Absolute Values}

We define a macro \verb!\abs{}! to invoke \texttt{amsmath}'s \verb!\lvert! and \verb!\rvert! commands for automatically sizing the left- and right-bars of an absolute value. Examples:
\begin{align}
	\abs{x} &&
	\abs{\frac{x}{y}} &&
	\abs{\int \D{x} \, e^{i px}} &&
	\int \D{x} \, \abs{e^{i px}}
	\ .
\end{align}
If you want to be fancier, you can use \texttt{mathtools}\footnote{\url{https://tex.stackexchange.com/a/35585/8094}} to define custom delimiters. For example:
\verb!\DeclarePairedDelimiter{\abs}{|}{|}! \, .


\subsection{Vector notation}

\begin{itemize}
	\item Vector $\vec{v}$ and dual vector $\row{w}$. Also $\ket{v}$ and $\bra{w}$.

	\item Transpose: The \acro{ISO}~80000 standard has suggestions. $A^T$ vs $A^\top$ vs $A^{\trans}$. I personally prefer $A^\text{T}$.
	
	\item I have an \verb!\aij{}{}! macro for tensors. For example: \verb!M\aij{i}{j}! gives $M\aij{i}{j}$.

\end{itemize}


\subsection{Miscellaneous}

\begin{itemize}
	\item Use $\mid$ instead of pipe for conditions: $p(x\mid y)$ versus $p(x|y)$.
	% 
	\item \textbf{Textual subscripts}: sometimes you have a subscript that is not an index, but shorthand for something textual. For example, the Green's function with Dirichlet boundary conditions is $G_\textnormal{D}$. The subscript is upright, not italicized, $G_D$. Use the \texttt{textnormal} command rather than the \texttt{text} command since this will automatically use the correct size. Comparison: $G_\textnormal{D},\, G_\text{D}$.
	% 
	\item Arrows with text under them: \texttt{xrightarrow} in the \texttt{amsmath} package. The square bracketed argument is under, the curly bracketed argument is over. Example: $\xrightarrow[\textnormal{low}]{\textnormal{hi}}$.
	% 
	\item We can use the $\defeq$ symbol, defined as a macro\footnote{\url{https://tex.stackexchange.com/a/4881/8094}} \verb!\defeq!, to denote assignment . The macro typesets the symbol so that the dots are the same size and aligned with the lines. In pedagogical writing, it may be useful to distinguish between equality $=$, assignment $\defeq$, and tautology $\equiv$. At least that is how I use these.\footnote{Arnold Arons brings up equal signs in \emph{Teaching Introductory Physics}, Chapter~3.23. I prefer $\defeq$ for definitions because it shows the asymmetry of the relation. The statement $a\defeq b$ means that $a$ is defined to be $b$. The equal sign $=$ is symmetric in appearance and in meaning: $a$ and $b$ are the same. I use  $a\equiv b$ sparingly to mean $a$ and $b$ are \emph{obviously} the same but in a way that is not necessarily derived mathematically. }
	% 
 \end{itemize}



\section{Space}

\subsection{Kerning: spacing between characters}
Math operators have a natural spacing before and after depending on the context. In the following example, spaces indicate that the coefficient $a$ multiplies the logarithm of $b$:
\begin{align}
	a\log b && a\log(bc)
\end{align}
The space on either side of $\log$ indicate that $\log$ is a mathematical operator. The second example still has the space between $a$ and $\log$, but has no space between $\log$ and $(bc)$ because the parenthesis belongs to the mathematical function.\footnote{Example from \url{https://tex.stackexchange.com/a/140647}}
% 
You can use \verb!\DeclareMathOperator! for functions that are not built in.

Using \verb!\left(! and \verb!\right)! will automatically size parentheses, but can mess up kerning:
\begin{align}
	f(x)
	f\left(x\right)
	f{\left(x\right)}
	&
	&
	\cos(x)
	\cos\left(x\right)
	\cos{\left(x\right)}
	{\cos}{\left(x\right)}
	\ .
\end{align}
The crude way to fix this is to put braces around the parentheses: \verb!{\left(x\right)}!.\footnote{\url{https://tex.stackexchange.com/a/2610/8094}}. However, if the parenthesis is attached to a mathematical operators, the operator must also be surrounded by braces: \verb!{\cos}!. In the examples above, you can see the effect of the braces on the spacing on either side.



\subsection{Manual Spacing}

\LaTeX{} has commands for manually inserting spacing: 
\begin{itemize}
	\item \verb!\,! thin\,space
	\item \verb!\:! medium\:space
	\item \verb!\;! thick\;space
	\item \verb$\!$ thin\!negative\!space.
		  %% note the use of a different symbol in \verb 		
\end{itemize}
The negative space can be helpful for manually adjusting the kerning for large parentheses raised to a power:
\begin{align}
	\left(\frac{\pi}{\sum_{i=1}^n x^i}\right)^{d+1}
	&&
	\left(\frac{\pi}{\sum_{i=1}^n x^i}\right)^{\!d+1}
	\ .
\end{align}
On the right we use \verb$\right)^{\!d+1}$.



\subsection{Ties create non-breaking spaces}

\LaTeX{} interprets periods as full stops (end of a sentence). It places extra space after the full stop. Use a \textbf{normal space} right after the period, \verb!.\ ! (``slash space'') to tell \LaTeX{} that a period is not a full stop and that it should insert insert a normal space not a double space.\footnote{The double space is sometimes considered old fashioned. You can use \texttt{frenchspacing} in your document to turn off the double space after a full stop.} Mr.\ Roboto versus Mr. Roboto. 


% Use a \textbf{tie} (tilde) when a period is not a full stop: Mr.~Roboto versus Mr. Roboto. 
A related construction is a \textbf{tie} (tilde). This gives \textbf{non-breaking space} which is a normal space that \LaTeX{} interprets as `part of the word'. This means that line breaks should not occur along the non-breaking space.
% 
You can also use ties this whenever you want to prevent a line break between words. \LaTeX{} interprets the tied words as a single word.
% 
Ties are also standard for citations: \verb!Tanedo et al.~\cite{citation}!. 


\subsection{Underlined space}

An example that comes up surprisingly often is the command \verb!\textvisiblespace! which produce an explicit space character:\textvisiblespace.

There is a neat macro one can define, \verb!\Vtextvisiblespace[1em]! that lets you write longer visible spaces:\Vtextvisiblespace[1em] and \Vtextvisiblespace[2em].

\section{Some Best Practices}

\subsection{\texorpdfstring{\LaTeX{} in a title}{LaTeX in a title}}

When using \LaTeX{} code in a section title, use the \texttt{texorpdfstring} command to define an \acro{UTF-8} string that the pdf can use for bookmarks. If you do not do this, there are annoying compilation warnings.


\subsection{Ranges}

Hyphens, en dashes, em dashes, and minus signs in math mode are all grammatically different.
\begin{itemize}
	\item En dashes replace hyphens in a compound adjective where one of the elements is a two-word compound: `post--Cold War era.'\footnote{From \url{https://www.merriam-webster.com/words-at-play/em-dash-en-dash-how-to-use}}
	\item En dashes are used for combinations of two names in place of the word `and.' For example, Randall--Sundrum model.
	\item For compound names of a single person, use a hyphen: Levi-Civita.
	\item A minus sign should be typeset in math mode, $-1$.
\end{itemize}
The choice between a hyphen and en dash can be tricky. For example \acro{APS} seems to prefer ``anti--de Sitter'' with an en dash, whereas others prefer a hyphen.\footnote{See 1 Jan '23: \url{https://en.wikipedia.org/wiki/Talk:Anti-de_Sitter_space}.}


\subsection{References}

Use \BibTeX\xspace. There are any number of \BibTeX\xspace managers. One that I like is Yuji Tachikawa's \texttt{spires.app}\footnote{\url{https://member.ipmu.jp/yuji.tachikawa/spires/}}; it links directly to the inSpire \acro{HEP} database. For styling, I recommend Jacques Distler's \texttt{utcaps.bst} which is a nice format that automatically inserts a hyper link to the \texttt{arXiv} version of a paper. If yo are fancy you may use {\rm B\kern-.05em{\sc i\kern-.025em b}\LaTeX{}. 


Standard citation managers with \BibTeX\xspace capability make a big deal about assigning unique \BibTeX\xspace citation keys to each reference. This can be tedious if you have to collaborate with someone else who has made up their own citation keys. Fortunately, in my field there is a single recognized database, inSpire\footnote{\url{https://inspirehep.net}; see also \acro{NASA/ADS} for astronomy.}, that assigns a unique \BibTeX entry for papers.  This means that it is good practice to select keys as follows:
\begin{itemize}
	\item If it exists, use the inSpire key. Tools like \texttt{spires.app} default to this key.
	\item If inSpire does not have the reference, use \acro{NASA/ADS}.
	\item If neither of those databases has the references, find a simple algorithm that all of your coauthors can agree upon. 
\end{itemize}
Items in the third category are usually books and websites. 



\section{Neat examples}

Suppose you would like to repeat and equation reference,
\begin{align}
	\e^{\I\pi} &= -1 
	\ .
	\label{eq:e:ipi}
\end{align}
Remember that equation above? Let's write it again,
\begin{align}
	\e^{\I\pi} &= -1 
	\ .
	\tag{\ref{eq:e:ipi}}
\end{align}
Observe that these have the same equation number. Instead of \texttt{label} we use \texttt{tag} with argument \texttt{ref}.

% %!TEX root = paper.tex
%% Update the above with the appropriate root

\section{Teaching Examples}

For lecture notes it is useful to have some framed environments to highlight examples and exercises. In the past I have used \texttt{framed} and \texttt{mdframed}. I think \texttt{tcolorbox} may be the best option now.

% \begin{tcolorbox}
% 	⟨environment content⟩
% \end{tcolorbox}

\begin{theorem}[Flip's Theorem]
This is an example theorem.
\end{theorem}

\begin{example}[Flip's Example]
This is an example of an example
\end{example}


\begin{exercise}[Solving a differential equation]
  Solve the following differential equation.
  \label{ex:solve:ode}
\end{exercise} 
\noindent Can you solve Exercise~\ref{ex:solve:ode}?


\begin{bigidea}[environments]
  With \texttt{tcolorbox}, one may `dress' existing environments in boxes. The call to the environments is unchanged.
  \label{idea:environments}
\end{bigidea} 
\noindent Refer to \bigidearef{}~\ref{idea:environments}.
% %!TEX root = paper.tex
%% Update the above with the appropriate root

\section{Code example}

These are examples of the \texttt{listings} package for typesetting code. See Overleaf\footnote{\url{https://www.overleaf.com/learn/latex/Code_listing}} and \acro{TeX.SE}~\footnote{\url{https://tex.stackexchange.com/a/350242}} for more information. By the way, printed out code is called `listings' because old computer languages has a \texttt{LIST} command to print out the numbered source code lines.\footnote{\url{https://softwareengineering.stackexchange.com/a/289729}}

\subsection{Inline}

If you just want to be able to insert \LaTeX{} into a document like this, you can use \verb!\verb!\footnote{\url{https://stackoverflow.com/a/66115768/16426341}}. The way it works is that you write \verb!\verb#TEXT#! where \verb!#! is any character that is not in \texttt{TEXT}. You can use listings package similarly with \verb!\lstinline! in place of \verb!\verb!.


\subsection{Jupyter Notebook}

Jupyter formatting from user \texttt{yogabonito} on \acro{TeX.SE}\footnote{\url{https://tex.stackexchange.com/a/350242/8094}}.

\begin{pyin}%[pyin01]
print("Hello world")
\end{pyin}
%  
\begin{pyprint}
Hello world
\end{pyprint}
% 
You get a `multiply defined label' warning if you do not explicitly label each of your Python inputs. 
%% EXAMPLE: 
% \begin{pyin}
% print("Hello world, too")
% \end{pyin}
% \begin{pyin}
% print("Hello world, three.")
% \end{pyin}
%% This gives two Python inputs with label `' (blank)
%% and thus returns a compiler warning.
% 
% 
Here we have a return value in the last line of the input cell.
\begin{pyin}[labelOfTheSecondInput]
def twicify(arg):
    print("Received:", arg, "- Will double now...")
    return 2 * arg
twicify(1)
\end{pyin}

\begin{pyprint}
Received: 1 - Will double now...
\end{pyprint}

\begin{pyout}
2
\end{pyout}
% 
% \subsection{Referencing input}
You can also reference the labeled input \ref{labelOfTheSecondInput}, from above.
% \begin{pyin}[anotherlabel]
% "and the counter will automatically do the right thing :)"
% \end{pyin}
% \begin{pyout}
% 'and the counter will automatically do the right thing :)'
% \end{pyout}


\subsection{\texorpdfstring{\LaTeX}{LaTeX}}

From user \texttt{hair-splitter} on \acro{TeX.SE}\footnote{\url{https://tex.stackexchange.com/a/637305/8094}}:

\begin{lstlisting}[style=latexstyle]
\documentclass{article}
\usepackage[T1]{fontenc}
\newcommand*{\mycommand}{Hello World!}
\begin{document}
  \mycommand
\end{document}
\end{lstlisting}


% %!TEX root = paper.tex
%% Update the above with the appropriate root

\section{\texorpdfstring{\LaTeX{} Style}{LaTeX Style}}

There is not a definitive \LaTeX{} style guide analogous to \acro{PEP-8}. However, I do have my own set of preferences. Here style refers to how the \LaTeX{} source files are written. Two \texttt{tex} files may produce identical \texttt{pdf} outputs but be stylistically different. A well styled document is that is as easy as possible to parse and edit as a human being. I have done my best to keep the source files for \emph{this} template well styled.


\subsection{Idiosyncracies}

While some coding style guides require a fixed width for the document. This gives meaning to a line of code and makes the resulting source more readable by avoiding unintentional text wrapping. \LaTeX{} is a bit different in that it is a typesetting language that is meant to handle paragraphs of text. Because modern editors naturally have text wrapping options, I do not feel strongly about enforcing a document-wide character width limit. Paragraphs of text should be allowed to wrap if that makes sense. However, mathematics environments should strive to make use of white space in service to readability.


\subsection{Spacing and Indents}

White space helps distinguish the document structure. 

\begin{enumerate}
	\item Sections should have three empty lines between each other.
	\item Sub-sections should have two empty lines between each other.
	\item All other units of paragraphs should have one empty line space between them.
\end{enumerate}
One may use with commented out empty lines to separate sentences from one another. 
% 
	This produces no paragraph break between the sentences in the output, but can help separate different ideas within a paragraph.
% 
	Similarly, one can combine this with indents to help visually organize the logical flow of a paragraph.


\subsection{Mathematics}

Use comments and white space to separate mathematics environments from plain text.
% 
The contents of a mathematics environment should use ample white space to separate each mathematical object as if these were words in a sentence.
% 
\begin{lstlisting}[style=latexstyle]
\begin{align}
  S_{\textnormal{fix}}^{\textnormal{Bulk}}
  & =
  \frac{-1}{g^2} 
  % \int d^{d+1} x  
  \int \D[d+1]{x}  
  \left( \frac{R}{z} \right)^{\!d-3}
  \frac{1}{2\xi}
  \left[
    \partial_\mu A^\mu
    -
    \xi\left( 
      z^{d-3} 
      \partial_z \left( \frac{A_z}{z^{d-3}} \right)
      -
      \left( \frac{R}{z} \right)^{\!2} 
      g^2 v(z) \, \pi
    \right)
  \right]^2 \ ,
\label{eq:SGFBulk}
\end{align}
\end{lstlisting}
% 
This produces the following:
\begin{align}
	S_{\textnormal{fix}}^{\textnormal{Bulk}}
	& =
	\frac{-1}{g^2} 
	% \int d^{d+1} x  
	\int \D[d+1]{x}  
	\left( \frac{R}{z} \right)^{\!d-3}\frac{1}{2\xi}
	\left[
	    \partial_\mu A^\mu
	    -
	    \xi\left( 
	        z^{d-3} \partial_z \left(\frac{A_z}{z^{d-3}}\right)
	        -
	        \left(\frac{R}{z}\right)^{\!2} g^2 v(z)\, \pi
	    \right)
	\right]^2 \ ,
% \label{eq:SGFBulk}
\end{align}
For long expressions, each line should be a well-defined `unit' of the mathematical expression. When there are multiple `words' in an expression, add white space to delimit them. The use of indentation to group elements of the same level should be self explanatory.

For example, simple fractions do not need any white space, while more complicated fractions should provide some help:
% 
\begin{lstlisting}[style=latexstyle]
\frac{ 
	b_\mathcal{O}
}{
	\Lambda^{\Delta_\mathcal{O} - \frac{d}{2} - 1 }
} 
\end{lstlisting}
% 
This may seem like overkill, but in an expression with multiple fractions it is helpful to be able to quickly visually parse each piece of the expression.


\subsection{When (not) to use macros}

Use macros (\texttt{newcommand} or \texttt{renewcommand}) when there is an expression that you use often \emph{and that you may want to change in the future}. Perhaps you have a variable that needs to be used uniformly across a document, but whose specific symbol may change. Use a macro and make it easy to change that symbol without doing a find-and-replace. Macros are also useful for making \LaTeX{} source more readable by truncating a tedious series of commands. 

However, do not use a macro just because it will make your life easier in the moment. One good reason \emph{not} to use a macro is as a shortcut for defining an environment. There is a `old style' of source preparation where someone will define macros:
% 
\begin{lstlisting}[style=latexstyle]
\newcommand{\be}{\begin{equation}}
\newcommand{\ee}{\end{equation}}
\end{lstlisting}
% 
This seems like a shortcut because it saves you the trouble of writing
% 
\begin{lstlisting}[style=latexstyle]
\begin{equation}
	...
\end{equation}
\end{lstlisting}
% 
There is a narrow range of tech-savvy for which this `trick' is useful: one has to be sophisticated enough where typing in six characters rather than thirty-one characters will save significant time, but one must also be na\"ive enough to not use a text editor that can do (1) text expansions or (2) context-aware text highlighting. 
% 
This latter point is what can drive a collaborator crazy: some \LaTeX{} editors look for \textbackslash\texttt{begin}--\textbackslash\texttt{end} pairs to identify math mode and highlight text in a helpful way. Macros like \texttt{be} and \texttt{ee} screw this up and can lead to linter warnings. In summary, do not define macros to simplify environments.


\subsection{Labels}

Labels should be descriptive, even if that means that they labels become a bit lengthy. Each label should start with some indication of what it is labeling, \texttt{eq} for equation or \texttt{fig} for figure, for example. Use colons to separate words.
% 
\begin{lstlisting}[style=latexstyle]
\label{eq:Euler:formula}
\label{sec:introduction}
\label{sec:introduction:past:work}
\end{lstlisting}
% 
Many \LaTeX{} editors have intelligent autocomplete that makes it easy to insert references to past labels, so it does not take any more time to have a lengthy label. It does, however, save time if your labels are descriptive and easy to select from a drop-down menu. 


\subsection{Collaborative writing}

A key principle of \LaTeX{} source should be making the document collaboration-friendly. It is \emph{not} sufficient that your \LaTeX{} source compiles. Your code needs to be:
\begin{enumerate}
	\item \textbf{Readable}. Use white space in a consistent way to reflect the underlying structure of your document and your expressions. Use consistent spacing between sections and subsections. Use ample white space within an equation to make each piece easy to identify. 
	\item \textbf{Editable}. Collaborators should be able to fix typos without having to do a deep dive of your source ode. 
\end{enumerate}
Here's what you are allowed to sacrifice in order to meet these goals:
\begin{enumerate}
	\item Your source code does not need to be short. Nobody will print out our source file. An equation that typesets to a single line can be spread out over a dozen lines if it helps the reader parse the \LaTeX{}. 
	\item Your source code does not need to be confined to a single file. 
	\item Your figures should be in a separate folder. In this template we use:
% 
\begin{lstlisting}[style=latexstyle]
\graphicspath{{figures/}}
\end{lstlisting}
% 
	\item Your pose writing---that is, writing that is not in math mode---can also be broken up using empty lines (comment out the line to avoid starting a new paragraph) and white space in order to elucidate the structure.\footnote{This can be useful if you tend to write convoluted sentences. You can `diagram' your sentence so that you see how each clause fits together.}
\end{enumerate}

A good way to collaborate is to use \texttt{git}/\texttt{github} or {Overleaf}. Overleaf is fantastic for abstracting away all version control. However, savvy users with an Overleaf membership can connect to Overleaf documents using \texttt{git}. This gives you the best of both worlds: you may use your favorite \acro{IDE} or \LaTeX{} editor and your favorite \texttt{git} client to synchronize to an Overleaf document that your collaborators can edit through the web interface if they wish.\footnote{You may notice that there is a \texttt{gitignore.txt} file in this template. It is a copy of the \texttt{.gitignore} file used by \texttt{git}. The \acro{macOS} interface does not like it when users try to manipulate files named dot-something since it worries that you might be messing up an important system file. Thus I have included a text file with the contents of \texttt{.gitignore} for users who simply want to copy this template folder using \acro{macOS}. You should then rename \texttt{gitignore.txt} to \texttt{.gitignore} if you are using \texttt{git}. Alternatively, download this file as a template from \texttt{github} and ignore \texttt{gitignore.txt} altogether.  }

\subsection{\texorpdfstring{\LaTeX{} Style Guide}{LaTeX Style Guide}}

\begin{itemize}
	\item Use \texttt{hyperref}. Modern documents should be internally and externally hyperlinked. Clicking on a reference to an equation should bring the reader to that equation. Use a bibliography style that allows hyperlinks to the \texttt{arXiv}.
	\item In \emph{informal} documents like this, I refer to websites (usually Stack Exchange) with a footnote and a hyperlink:
	% 
\begin{lstlisting}[style=latexstyle]
...a great website.%
\footnote{\url{https://www...}}
\end{lstlisting}
	% 
	This avoids having to use \BibTeX for one-off references and makes the reference easily clickable. 
	% 
\end{itemize}


\subsection{Copy Editing Style Guide}

These are non-specific style points:

\begin{itemize}
	\item Footnotes go after punctuation.%
			\footnote{Like this.}
	\item Use whitespace to separate footnotes:
	% 
\begin{lstlisting}[style=latexstyle]
We now make an important%
  \footnote{
	Observe the comment and whitespace.
	There is no additional spacing between
	`important' and this footnote. }
point about style.
\end{lstlisting}
	%
	\item You may consider using \textbackslash\texttt{frenchspacing} in your document.\footnote{Hat tip to Eddie Kohler, \url{https://www.read.seas.harvard.edu/~kohler/latex.html}}
\end{itemize}


\subsection{Breaking the rules}

It is okay to break the style rules when doing so supports clarity. Sometimes it makes sense to throw in lots of extra white space to separate ideas.


\subsection{Cleaning Up}

Sometimes it is okay to leave a mess. You may want to preserve an old `verbose' version of your source files that has intermediate steps spelled out overly-pedagogically. You may find it useful to leave comments in the source files to remind yourself of the structure of your argument. These changes are preserved if you use version control, but sometimes you want the redundancy and convenience of having old text readily available. Just remember to go through your source code carefully to strip it of comments before submitting to a public repository like the \texttt{arXiv}.
% %!TEX root = paper.tex
%% Update the above with the appropriate root

\section{References}

\subsection{\texorpdfstring{\LaTeX{} Style}{LaTeX Style}}

\begin{itemize}
    \item Didier Verna, ``Towards \LaTeX{} coding standards.\footnote{\url{https://tug.org/TUGboat/tb32-3/tb102verna.pdf}; video: \url{http://zeeba.tv/toward-latex-coding-standards/}}'' 
    \item See also Philippe Beliveau's summary of Verna's piece.\footnote{\url{https://medium.com/@pbeliveau/latex-coding-standards-f82743b7866b}}
    \item Evan Chen, ``Evan's \LaTeX{} Style Guide.\footnote{\url{https://web.evanchen.cc/latex-style-guide.html}}''
    \item Eddie Kohler, ``LaTeX Usage Notes.\footnote{\url{https://www.read.seas.harvard.edu/~kohler/latex.html}}''
\end{itemize}


\subsection{Style Guides}

Most publications have a style guide, analogous to the well-known \acro{APA} style in social sciences. Checking for strict adherence to those guides have historically been the role of copy editors, though the number of published papers continues to grow with no obvious increase in resources for copy editing. 

\begin{itemize}
    \item There is an \acro{ISO} standard for typesetting mathematics and physics. As of 2023 it is \acro{ISO80000}.
    \item Strunk \& White, \emph{The Elements of Style}
    \item The \emph{Review of Modern Physics} style guide. 
\end{itemize}

\subsection{Typography References}

The standard typography reference is Robert Bringhurst's \emph{The Elements of Typographic Style}. 
The following references focus specifically on typography and \LaTeX{}.
\begin{itemize}
    \item Consistent typography on \acro{TeX.SE}\footnote{\url{https://tex.stackexchange.com/questions/29840/consistent-typography}}
    \item List of best practices references on \acro{TeX.SE}\footnote{\url{https://tex.stackexchange.com/questions/577/best-practices-references}}, including the list of obsolete packages and commands in \LaTeX{}~2e\footnote{\url{https://www.ctan.org/tex-archive/info/l2tabu/english/}}
    \item ``The Art of \LaTeX{},'' a list of guidelines from Fan Pu Zeng\footnote{\url{https://fanpu.io/blog/2023/latex-tips/}}
    \item Showcase of beautiful typography in \LaTeX{}\footnote{\url{https://tex.stackexchange.com/q/1319/8094}}
\end{itemize}



%% CHAPTER SUBAPPENDIX %% require report class
\begin{subappendices}
\section{Subappendix}\label{sec:subappendix:eg}
This chapter has its own special appendix.
\end{subappendices}



\section*{Acknowledgments}

\acro{PT}\ thanks 
\emph{your name here}
for useful comments and discussions. 
%
\acro{PT} thanks 
    the Aspen Center for Physics (\acro{NSF} grant \acro{\#1066293})
    % and the Kavli Institute for Theoretical Physics (\acro{NSF} grant \acro{PHY-1748958})`'
    for 
    its 
    % their
    hospitality during a period where part of this work was completed. 
%
% \acro{PT} is supported by the \acro{DOE} grant \acro{DE-SC}/0008541.
\acro{PT} is supported by a \acro{NSF CAREER} award (\#2045333).

%% Appendices
\appendix
\chapter{Proper appendix}
Unlike Chapter~\ref{sec:subappendix:eg}, this is an appendix at the end of the document rather than a sub-appendix within a chapter. Check out the index that follows this chapter.

\section{Things to work on}

It may be nice to incorporate something like \texttt{classicthesis}\footnote{\url{https://www.ctan.org/tex-archive/macros/latex/contrib/classicthesis/}}


\printindex

%% No bibliography down here, see \addbibresource near header

\end{document}